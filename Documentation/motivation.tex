% Conduct a motivating experiment that explains
%     Why is this problem important
%     Assuming caches are always good
%         Why multi tier
%         Why partitioning
%             Consider ~10-100 VMs on a single host
%             Global cache is not efficient

\section{Motivation}
The motivation for this work are two-fold. First, we show why we need a smart partitioning algorithm. Second, We show why we need a multi layer cache

The need for partitioning the SSD cache has been studied in [vCacheShare, ...]. To illustrate we conduct a simple experiment where two VMs, namely VM1 and VM2, residing in the same host share a caching device. From figure XX, one can see that VM1 has a high random I/O with a very low hit ratio while VM2 has a low I/O but has a high hit rate. Figure YY, shows that despite VM2 could benefit largely from a cache, it has a very low space allocation in the cache, due to contention from VM1. From figure XX and YY, one can infer that contention can make a cache entirely unusable.

The case for multi-tier cache:
\begin{itemize}
\item Some VMs have higher priority than others.
\item Latency sensitive VMs can greatly benefit from a multi-tier cache.
\item Interactive VMs that has a very low I/O, but a very high hit rate can be ``separated" and placed in the lowest latency cache for high reponsiveness.
\end{itemize}
